\usepackage{listings}
\usepackage[english]{babel}
\usepackage[utf8]{inputenc}
\usepackage[dvips]{graphicx}
\usepackage{amsmath,amsthm,amssymb,tipa,dsfont,mathtools,mathrsfs,here,titlesec,fancyhdr}
\usepackage{anysize}
\usepackage{subfigure}
\usepackage{color}
\usepackage{enumerate}
\usepackage{booktabs}
\usepackage{rotating}
%\usepackage{parskip} % to avoid // 
\usepackage[hidelinks]{hyperref}
\usepackage{url}
\usepackage{multirow}
\renewcommand{\qedsymbol}{\rule{0.7em}{0.7em}}
\usepackage{comment} % begin{comment} to comment large sections
\usepackage[font=scriptsize]{caption}
\usepackage{rotating} % to rotate tables 
\usepackage{tikz}
\usetikzlibrary{shapes,decorations,arrows,calc,arrows.meta,fit,positioning}
\tikzset{
	-Latex,auto,node distance =1 cm and 1 cm,semithick,
	state/.style ={ellipse, draw, minimum width = 0.7 cm},
	point/.style = {circle, draw, inner sep=0.04cm,fill,node contents={}},
	bidirected/.style={Latex-Latex,dashed},
	el/.style = {inner sep=2pt, align=left, sloped}
}

\definecolor{Grey}{RGB}{150, 150, 150}
\definecolor{PPblue}{RGB}{0,114,198}
\definecolor{VOXblue}{RGB}{0,114,198}
\definecolor{PSOEred}{RGB}{232,0,0}
\definecolor{UPred}{RGB}{232,0,0}
\newcommand{\sym}[1]{\rlap{#1}}

\definecolor{mypink}{rgb}{0.858, 0.188, 0.478}
\definecolor{myorange}{rgb}{1.0, 0.49, 0.0}
\definecolor{mypurple}{rgb}{0.6, 0.4, 0.8}
%\usepackage[colorlinks=true,linkcolor=blue,citecolor=blue,hyperfootnotes=false]{hyperref} 
\hypersetup{
	colorlinks,
	citecolor=blue,
	linkcolor=mypink,
	urlcolor=mypurple}
%Els comandaments següents són per a linkejar (han d'estar al final de tots els usepackage)
%\usepackage[colorlinks]{hyperref}%aquests paquets s'utilitzen per a poder linkejar coses.
%\hypersetup{citecolor=red}
%\hypersetup{linkcolor=red}
%\hypersetup{urlcolor=red}
%\usepackage{cleveref}%aquest també.

\usepackage[T1]{fontenc} % for porper quotation marks 
\PassOptionsToPackage{svgnames}{xcolor}
\usepackage{pgfplots}


\usepackage{tcolorbox}
\usepackage{lipsum}
\tcbuselibrary{skins,breakable}
\usetikzlibrary{shadings,shadows}
\newenvironment{myblock}[1]{%
	\tcolorbox[beamer,%
	noparskip,breakable,
	colback=LightBlue,colframe=DarkBlue,%
	colbacklower=DarkBlue!75!LightBlue,%
	title=#1]}%
{\endtcolorbox}

\usepackage{tikz}
\usetikzlibrary{shapes,decorations,arrows,calc,arrows.meta,fit,positioning}
\tikzset{
	-Latex,auto,node distance =1 cm and 1 cm,semithick,
	state/.style ={ellipse, draw, minimum width = 0.7 cm},
	point/.style = {circle, draw, inner sep=0.04cm,fill,node contents={}},
	bidirected/.style={Latex-Latex,dashed},
	el/.style = {inner sep=2pt, align=left, sloped}
}














\renewcommand{\baselinestretch}{1.2} %separació entre linies
\marginsize{2.3cm}{2.3cm}{1cm}{2cm} %Margens


%ací definim els colors que anem a gastar
\definecolor{secction}{rgb}{0.62,0.31,0.00}
\definecolor{subsection}{rgb}{0.33,0.00,0.33}
\definecolor{mygreen}{RGB}{28,172,0}


%S'utilitza per a insertar programes de forma més professional
%\lstset{language=Matlab,numbers=left,frame=single,title=\lstname}
\lstset{language=Matlab,%
    %basicstyle=\color{red},
    breaklines=true,%
    morekeywords={matlab2tikz},
    keywordstyle=\color{blue},%
    morekeywords=[2]{1}, keywordstyle=[2]{\color{black}},
    identifierstyle=\color{black},%
    %stringstyle=\color{mylilas},
    commentstyle=\color{mygreen},%
    showstringspaces=false,%without this there will be a symbol in the places where there is a space
    numbers=left,%
    numberstyle={\tiny \color{black}},% size of the numbers
    numbersep=9pt, % this defines how far the numbers are from the text
    emph=[1]{for,end,break},emphstyle=[1]\color{blue}, %some words to emphasise
    %emph=[2]{word1,word2}, emphstyle=[2]{style},
}

\usepackage[colorinlistoftodos,textwidth=3cm]{todonotes}
\newcommand{\todoINFO}[1]{\todo[color=blue!25]{INFO: #1}}
\newcommand{\todoIMPORTANT}[1]{\todo[color=red!25]{IMPORTANT: #1}}
\newcommand{\todoREV}[1]{\todo[color=green!25]{REVIEWED: #1}}
%Aquí redefinimos los comandos teorema, nota, etc... para que sea más fácil de escribir. Lo que está entre corchetes "[,]" es para que enumere los teoremas en función de la sección, si se quita, lo enumera sobre el total.
\theoremstyle{plain}
\newtheorem{teo}{Theorem}[section]
\newtheorem{prop}{Proposition}[section]
\newtheorem{exe}{Exercise}
\theoremstyle{definition}
\newtheorem{defi}{Definition}[section]
\newtheorem{nota}{Note}
\DeclarePairedDelimiter\abs{\lvert}{\rvert}%
\DeclarePairedDelimiter\norm{\lVert}{\rVert}%
%abreviatures dels comandaments "begin/end". Es una chorradita por no escribirlo todo, puedes usarlo o no.
\newcommand{\be}{\begin{exe}}
\newcommand{\ee}{\end{exe}}
\newcommand{\bt}{\begin{teo}}
\newcommand{\et}{\end{teo}}
\newcommand{\bd}{\begin{defi}}
\newcommand{\ed}{\end{defi}}
\newcommand{\bn}{\begin{nota}}
\newcommand{\en}{\end{nota}}
\newcommand{\bp}{\begin{proof}}
\newcommand{\ep}{\end{proof}}
\usepackage{accents}
\newcommand{\ubar}[1]{\underaccent{\bar}{#1}}

%abreviatures de símbols. Esto en realidad no lo utilizo nunca, pero si te acostumbras es más cómodo.
\newcommand{\e}{\exists}
\newcommand{\fa}{\forall}
%\newcommand{\iff}{\Leftrightarrow}

%Comandaments utilitzats per a donar un millor estil a les pàgines (la xorradeta de la linia de dalt)
\usepackage{fancyhdr}
\pagestyle{fancy}
\lhead{}
%\rhead{Luis Ignacio Menéndez García}
%\renewcommand{\footrulewidth}{0.3pt}

\renewcommand{\headrulewidth}{0pt}   % no line under the header
\renewcommand{\footrulewidth}{0pt}   % no line above the footer


%here we use a new command for the figures titles
\newcommand*{\figuretitle}[1]{%
	{\centering%   <--------  will only affect the title because of the grouping (by the
		\textbf{#1}%              braces before \centering and behind \medskip). If you remove
		\par\medskip}%            these braces the whole body of a {figure} env will be centered.
}

\DeclareMathOperator*{\E}{\mathbb{E}}
\DeclareMathOperator*{\R}{\mathbb{R}}
\DeclareMathOperator*{\Lag}{\mathscr{L}}
\DeclareMathOperator*{\contr}{\Rightarrow\Leftarrow}
\DeclareMathOperator*{\convprob}{\overset{p}{\to}}
\DeclareMathOperator*{\convas}{\overset{a.s.}{\to}}
\DeclareMathOperator*{\convd}{\overset{d}{\to}}
\DeclareMathOperator*{\impose}{\stackrel{!}{=}}



%change colorlinks=false if you don't wan't any hyperlinks to have color

%cmd+n to open a new wnvironment for equation see macros in options of texstudio
